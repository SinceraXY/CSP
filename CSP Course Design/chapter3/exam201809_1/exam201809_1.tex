\section{201809-1 卖菜}

\subsection{题目描述}%题目描述请保留在文档内
在一条街上有 n 个卖菜的商店,按 1 至 n 的顺序排成一排,这些商店都卖一种蔬菜。

第一天,每个商店都自己定了一个价格。店主们希望自己的菜价和其他商店的一致,第二天,每
一家商店都会根据他自己和相邻商店的价格调整自己的价格。具体的,每家商店都会将第二天的菜价
设置为自己和相邻商店第一天菜价的平均值(用去尾法取整)。

注意,编号为 1 的商店只有一个相邻的商店 2,编号为 n 的商店只有一个相邻的商店 n − 1,其他
编号为 i 的商店有两个相邻的商店 i − 1 和 i + 1。

给定第一天各个商店的菜价,请计算第二天每个商店的菜价。
\subsection{输入格式}

输入的第一行包含一个整数$n$,表示商店的数量。

第二行包含$n$个整数,依次表示每个商店第一天的菜价。

\subsection{输出格式}

输出一行,包含$n$个正整数,依次表示每个商店第二天的菜价。

\subsection{样例输入}

\begin{lstlisting}[numbers=none]
8
4 1 3 1 6 5 17 9
\end{lstlisting}

\subsection{样例输出}

\begin{lstlisting}[numbers=none]
2 2 1 3 4 9 10 13
\end{lstlisting}

\subsection{评测用例规模与约定}

对于所有评测用例,$2\leq n\leq1000$,第一天每个商店的菜价为不超过 10000 的正整数。

\subsection{对讲义中代码的点评或纠错}
清晰简单: 代码结构相对简单,易于理解。它使用了一个循环来计算移动平均值,然后输出结果。

使用了有意义的变量名: 变量名命名得相对清晰,例如,prices数组用于存储原始价格数据,newprices数组用于存储移动平均值。

使用常量: 代码中使用const int MAX = 1000;来定义了一个常量,以表示数组的最大长度,这有助于提高代码的可维护性。

\subsection{自己原创的代码的点评与注释}

首先读取商店数量和第一天的菜价,然后根据每个商店及其相邻商店的菜价,计算出第二天的菜价。最后,输出第二天每个商店的菜价。

\begin{lstlisting}[language=C++]
    #include <iostream>
    #include <vector>
    
    using namespace std;
    
    int main() {
        // 读取商店数量
        int n;
        cin >> n;
    
        // 读取第一天各个商店的菜价
        vector<int> prices(n);
        for (int i = 0; i < n; ++i) {
            cin >> prices[i];
        }
    
        // 计算第二天每个商店的菜价
        vector<int> nextDayPrices(n);
        for (int i = 0; i < n; ++i) {
            // 初始化平均值为当前商店的菜价
            int average = prices[i];
    
            // 如果不是第一个商店,加上左边商店的菜价
            if (i > 0) {
                average += prices[i - 1];
            }
    
            // 如果不是最后一个商店,加上右边商店的菜价
            if (i < n - 1) {
                average += prices[i + 1];
            }
    
            // 计算平均值并更新第二天的菜价
            nextDayPrices[i] = average / ((i > 0) + 1 + (i < n - 1));
        }
    
        // 输出第二天每个商店的菜价
        for (int i = 0; i < n; ++i) {
            cout << nextDayPrices[i] << " ";
        }
    
        return 0;
    }
    
\end{lstlisting}