\section{201812-1 小明上学}

\subsection{题目背景}

小明是汉东省政法大学附属中学的一名学生,他每天都要骑自行车往返于家和学校。为了能尽可能充足地睡眠,他希望能够预计自己上学所需要的时间。他上学需要经过数段道路,相邻两段道路之间设有至多一盏红绿灯。

京州市的红绿灯是这样工作的:每盏红绿灯有红、黄、绿三盏灯和一个能够显示倒计时的显示牌。假设红绿灯被设定为红灯$r$秒,黄灯$y$秒,绿灯$g$秒,那么从$0$时刻起,$[0,r)$秒内亮红灯,车辆不许通过;$[r, r+g)$秒内亮绿灯,车辆允许通过;$[r+g, r+g+y)$秒内亮黄灯,车辆不许通过,然后依次循环。倒计时的显示牌上显示的数字$l(l > 0)$是指距离下一次信号灯变化的秒数。

\subsection{问题描述}

一次上学的路上,小明记录下了经过每段路的时间,和各个红绿灯在小明到达路口时的颜色和倒计时秒数。希望你帮忙计算此次小明上学所用的时间。

\subsection{输入格式}

输入的第一行包含空格分隔的三个正整数$r$、$y$、$g$,表示红绿灯的设置。这三个数均不超过 $10^6$。

输入的第二行包含一个正整数 $n(n\leq 100)$,表示小明总共经过的道路段数和看到的红绿灯数目。

接下来的$n$行,每行包含空格分隔的两个整数$k$、$t$。$k=0$ 表示经过了一段道路,耗时$t$秒,此处$t$不超过 $10^6$;$k=1$、$2$、$3$ 时,分别表示看到了一个红灯、黄灯、绿灯,且倒计时显示牌上显示的数字是$t$,此处$t$分别不会超过$r$、$y$、$g$。
 
\subsection{输出格式}

输出一个数字,表示此次小明上学所用的时间。

\subsection{样例 1 输入}

\begin{lstlisting}[numbers=none]
30 3 30
8
0 10
1 5
0 11
2 2
0 6
0 3
3 10
0 3
\end{lstlisting}

\subsection{样例 1 输出}

\begin{lstlisting}[numbers=none]
70
\end{lstlisting}

\subsection{样例解释}

小明先经过第一段道路,用时 10 秒,然后等待 5 秒的红灯,再经过第二段道路,用时 11 秒,然后等待 2 秒的黄灯和 30 秒的红灯,再经过第三段、第四段道路,分别用时6、3秒,然后通过绿灯,再经过最后一段道路,用时 3 秒。共计 $10 + 5 + 11 + 2 + 30 + 6 + 3 + 3=70$ 秒。

\subsection{评测用例规模与约定}

测试点 1, 2 中不存在任何信号灯。
测试点 3, 4 中所有的信号灯在被观察时均为绿灯。
测试点 5, 6 中所有的信号灯在被观察时均为红灯。
测试点 7, 8 中所有的信号灯在被观察时均为黄灯。
测试点 9, 10 中将出现各种可能的情况。

\subsection{对讲义中代码的点评或纠错}
逻辑清晰:代码采用了简单的if-else语句来处理不同情况,逻辑相对清晰,易于理解。

变量命名:变量的命名相对清晰,例如r、y、g分别代表红灯、黄灯、绿灯的时间,n代表车辆数量等。

没有错误处理:代码假定输入是有效的,但没有进行错误处理。如果输入格式不符合要求,可能会导致程序崩溃或产生不正确的结果。
\subsection{自己原创的代码的点评与注释}

读取红绿灯的设置。

读取道路段数和红绿灯数目。

遍历每一段道路和红绿灯,根据不同的情况计算总时间。

输出总时间。

\begin{lstlisting}[language=C++]
    #include <iostream>
    using namespace std;
    
    int main() {
        // 读取红绿灯设置
        int r, y, g;
        cin >> r >> y >> g;
    
        // 读取道路段数和红绿灯数目
        int n;
        cin >> n;
    
        int totalTime = 0; // 记录总时间
    
        for (int i = 0; i < n; ++i) {
            int k, t; // k表示道路或红绿灯类型,t表示经过道路的时间或倒计时时间
            cin >> k >> t;
    
            if (k == 0) {
                // 经过道路的情况
                totalTime += t;
            } else {
                // 遇到红绿灯的情况
                if (k == 1) {
                    // 红灯,需要等待
                    totalTime += (r - t);
                } else if (k == 2) {
                    // 黄灯,需要等待
                    totalTime += (r + g - t);
                }
                // 绿灯无需等待,直接通过
            }
        }
    
        cout << totalTime << endl;
    
        return 0;
    }
    
\end{lstlisting}