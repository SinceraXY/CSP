\section{201609-1 最大波动}

\subsection{题目描述}%题目描述请保留在文档内
小明正在利用股票的波动程度来研究股票。小明拿到了一只股票每天收盘时的价格,他想知道,这
只股票连续几天的最大波动值是多少,即在这几天中某天收盘价格与前一天收盘价格之差的绝对值最
大是多少。
\subsection{输入格式}

输入的第一行包含一个整数$n$,表示小明拿到的收盘价格的连续天数。

第二行包含$n$个正整数,依次表示每天的收盘价格。

\subsection{输出格式}

输出一个整数,表示这只股票这$n$天中的最大波动值。

\subsection{样例输入}

\begin{lstlisting}[numbers=none]
6
2 5 5 7 3 5
\end{lstlisting}

\subsection{样例输出}

\begin{lstlisting}[numbers=none]
4
\end{lstlisting}

\subsection{样例解释}

第四天和第五天之间的波动最大,波动值为 |3 − 7| = 4。

\subsection{评测用例规模与约定}

对于所有评测用例,$2\leq n\leq1000$.股票每一天的价格为 1 到 10000 之间的整数。

\subsection{对讲义中代码的点评或纠错}
清晰简单: 代码结构相对简单,易于理解。主要使用了一个for循环来逐个读取输入的整数,并计算差值。

使用了有意义的变量名: 变量名命名得相对清晰,例如,n用于存储整数的数量,max用于存储最大的差值,pre和cur用于存储前一个和当前的整数。

使用标准库函数: 代码使用了abs函数来计算整数的绝对值,这是一个标准库函数,避免了手动编写绝对值计算的逻辑。

\subsection{自己原创的代码的点评与注释}

首先读取股票连续天数n,然后读取每天的收盘价格并存储在vector中。接下来,通过遍历计算每天与前一天的价格差的绝对值,找到最大波动值并输出。

\begin{lstlisting}[language=C++]
    #include <iostream>
    #include <vector>
    #include <cmath>
    using namespace std;
    
    int main() {
        // 读取股票连续天数
        int n;
        cin >> n;
    
        // 存储每天的收盘价格
        vector<int> prices(n);
        for (int i = 0; i < n; ++i) {
            cin >> prices[i];
        }
    
        // 计算最大波动值
        int maxFluctuation = 0;
        for (int i = 1; i < n; ++i) {
            // 计算每天与前一天的价格差的绝对值
            int fluctuation = abs(prices[i] - prices[i - 1]);
            // 更新最大波动值
            maxFluctuation = max(maxFluctuation, fluctuation);
        }
    
        // 输出最大波动值
        cout << maxFluctuation << endl;
        return 0;
    }
    
\end{lstlisting}