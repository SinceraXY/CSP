\section{202203-1 未初始化警告}

\subsection{题目背景}

\subsection{问题描述}

\subsection{输入格式}

\subsection{输出格式}

\subsection{样例 1 输入}

\subsection{样例 1 输出}

\subsection{样例解释}

\subsection{评测用例规模与约定}

\subsection{对讲义中代码的点评或纠错}
 
优点:

简洁明了: 代码结构简单,逻辑清晰,易于理解。主要使用了循环、条件语句和向量等基本结构,使得代码整体较为简洁。

使用标准库: 使用了 C++ 标准库中的 vector 容器,简化了对一系列元素的管理,提高了代码的可读性和可维护性。

良好的变量命名: 使用了有意义的变量名,如 initialized、$uninitialized\_count$ 等,有助于理解代码的含义。

对数组越界进行了处理: 在定义 vector<bool> initialized(n + 1, false) 时,考虑到数组索引从 1 开始,避免了数组越界的问题。

合理使用布尔类型: 使用布尔类型的向量,减小了内存占用,提高了空间效率。

\subsection{自己原创的代码的点评与注释}

数据结构选择: 使用一个布尔型的 vector(initialized)来记录每个变量是否已被初始化,通过下标与变量编号对应。

遍历赋值语句: 使用一个循环遍历每一条赋值语句,检查右值是否已经被初始化。

未初始化右值计数: 对于每一条赋值语句,如果右值是一个变量且未被初始化,就将未初始化右值的计数器加一。

更新左值状态: 每次处理完一条赋值语句后,将左值标记为已被初始化。

输出结果: 最后输出未初始化右值的赋值语句数量。

\begin{lstlisting}[language=C++]
    #include <iostream>
    #include <vector>
    #include <unordered_set>
    
    using namespace std;
    
    int main() {
        // 读入变量数量和赋值语句的条数
        int n, k;
        cin >> n >> k;
    
        // 初始化变量是否已被赋值的集合
        vector<bool> initialized(n + 1, false);
    
        // 初始化未初始化右值计数器
        int uninitializedCount = 0;
    
        // 处理每一条赋值语句
        for (int i = 0; i < k; ++i) {
            int xi, yi;
            cin >> xi >> yi;
    
            // 如果右值是变量,检查是否已被初始化
            if (yi > 0 && !initialized[yi]) {
                // 未被初始化,增加计数器
                uninitializedCount++;
            }
    
            // 将左值标记为已被初始化
            initialized[xi] = true;
        }
    
        // 输出未初始化右值的赋值语句条数
        cout << uninitializedCount << endl;
    
        return 0;
    }    
\end{lstlisting}