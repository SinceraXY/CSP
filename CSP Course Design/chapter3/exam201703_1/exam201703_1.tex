\section{201703-1 分蛋糕}

\subsection{题目背景}

\subsection{问题描述}

\subsection{输入格式}

\subsection{输出格式}

\subsection{样例 1 输入}

\subsection{样例 1 输出}

\subsection{样例解释}

\subsection{评测用例规模与约定}

\subsection{对讲义中代码的点评或纠错}
 
优点:

简洁而高效的算法: 代码使用了一次循环,通过累加元素值的方式高效地计算满足条件的分组数量,算法简洁而有效。

使用常量和数组: 使用了常量 maxn 来定义数组的最大长度,提高了代码的可维护性。同时,使用数组 a 存储输入元素,对于序列数据结构的处理更加方便。

避免了全局变量: 除了常量 maxn,所有变量都被限制在 main 函数的局部作用域内,避免了全局变量可能带来的问题。

使用 cur 优雅地跟踪当前累加和: 使用 cur 变量巧妙地跟踪当前累加和,提高了代码的可读性。

合理的输入处理: 通过使用循环读取输入,对于数组的初始化和元素的输入都采用了一致的方式。

清晰的逻辑结构: 代码逻辑清晰,易于理解。通过一个循环遍历数组,累加元素值,判断是否满足条件,并更新计数器,最后输出结果。

\subsection{自己原创的代码的点评与注释}

读入蛋糕的数量n和每个朋友至少要分到的蛋糕重量k。

读入n个蛋糕的重量。

对蛋糕重量数组进行排序。

遍历排序后的蛋糕数组,依次将蛋糕分给朋友,每次分给朋友后检查总重量是否达到k。

输出分到蛋糕的朋友数量。

\begin{lstlisting}[language=C++]
    #include <iostream>
    #include <vector>
    #include <algorithm>
    
    using namespace std;
    
    int main() {
        // 读入 n 和 k
        int n, k;
        cin >> n >> k;
    
        // 读入蛋糕的重量数组
        vector<int> weights(n);
        for (int i = 0; i < n; ++i) {
            cin >> weights[i];
        }
    
        // 对蛋糕重量数组进行排序
        sort(weights.begin(), weights.end());
    
        // 初始化朋友计数器和当前朋友所分蛋糕的总重量
        int friendCount = 0;
        int currentWeight = 0;
    
        // 遍历蛋糕,分给朋友
        for (int i = 0; i < n; ++i) {
            // 当前朋友分到的蛋糕总重量加上当前蛋糕的重量
            currentWeight += weights[i];
            
            // 如果当前蛋糕分给朋友后总重量大于等于 k,表示当前朋友分到蛋糕
            if (currentWeight >= k) {
                // 重置当前朋友分到的蛋糕总重量,为下一个朋友做准备
                currentWeight = 0;
                // 朋友计数器加一
                friendCount++;
            }
        }
    
        // 输出结果
        cout << friendCount << endl;
    
        return 0;
    }    
\end{lstlisting}