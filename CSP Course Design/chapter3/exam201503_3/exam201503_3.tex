\section{201503-3 节日}

\subsection{题目背景}

\subsection{问题描述}

\subsection{输入格式}

\subsection{输出格式}

\subsection{样例 1 输入}

\subsection{样例 1 输出}

\subsection{样例解释}

\subsection{评测用例规模与约定}

\subsection{对讲义中代码的点评或纠错}
 
优点:

使用结构体: 使用结构体Date将日期相关的信息封装在一起,提高了代码的可读性和可维护性。

成员函数的封装: isLeap() 和 daysInMonth() 函数封装了与日期相关的逻辑,使代码更加模块化。

明确函数的功能: 函数和成员变量的命名相对清晰,有助于理解代码的功能。

合理利用结构体成员: 使用结构体成员变量表示年、月、日和星期,使代码更直观。

使用 printf 进行格式化输出: 使用 printf 提高了输出格式的灵活性和可读性。

处理日期的逻辑清晰: addOneDay() 函数处理日期的逻辑相对清晰,考虑了月末和年末的进位。

bool isLastDay() 函数的逻辑非常简单,通过逻辑与运算符$\&\&$检查当前日期的月份是否为12(即12月),且日期是否为31日,这种简单直接的实现有助于减少代码的复杂性和错误的可能性。
并且这个函数体现了面向对象编程的封装性。通过将日期相关的逻辑封装在结构体内部,提高了代码的模块化和可维护性。

缺点和改进空间:

输入异常处理: 缺少对输入的合法性检查。应该考虑在读入a、b、c、y1和y2之前添加输入合法性的检查,以确保输入的有效性。

\subsection{自己原创的代码的点评与注释}

getFirstDayOfWeek 函数用于计算某年某月第一天是星期几, calculateDate 函数用于计算某年某月第 b 个星期 c 的日期。在 main 函数中,按照要求输出结果。

\begin{lstlisting}[language=C++]
    #include <iostream>
    #include <iomanip>
    
    using namespace std;
    
    // 判断是否是闰年的函数
    bool isLeapYear(int year) {
        if (year % 400 == 0 || (year % 4 == 0 && year % 100 != 0)) {
            return true;
        }
        return false;
    }
    
    // 计算某年某月第一天是星期几的函数
    int getFirstDayOfWeek(int year, int month) {
        // 已知 1850 年 1 月 1 日是星期二
        int days = 0;
        for (int i = 1850; i < year; ++i) {
            days += isLeapYear(i) ? 366 : 365;
        }
        for (int i = 1; i < month; ++i) {
            switch (i) {
                case 1: case 3: case 5: case 7: case 8: case 10: case 12:
                    days += 31;
                    break;
                case 4: case 6: case 9: case 11:
                    days += 30;
                    break;
                case 2:
                    days += isLeapYear(year) ? 29 : 28;
                    break;
            }
        }
        return (days + 2) % 7;  // 2 corresponds to the first day being Tuesday
    }
    
    // 计算某年某月第 b 个星期 c 的日期的函数
    string calculateDate(int year, int month, int b, int c) {
        int firstDayOfWeek = getFirstDayOfWeek(year, month);
        int targetDay = c - firstDayOfWeek + (b - 1) * 7;
    
        // 如果 targetDay 小于等于零或者大于该月的天数,表示不存在符合条件的日期
        if (targetDay <= 0 || targetDay > 31) {
            return "none";
        }
    
        // 如果 targetDay 大于该月的天数,取最后一天
        int daysInMonth;
        switch (month) {
            case 1: case 3: case 5: case 7: case 8: case 10: case 12:
                daysInMonth = 31;
                break;
            case 4: case 6: case 9: case 11:
                daysInMonth = 30;
                break;
            case 2:
                daysInMonth = isLeapYear(year) ? 29 : 28;
                break;
        }
    
        if (targetDay > daysInMonth) {
            targetDay = daysInMonth;
        }
    
        // 格式化输出日期
        stringstream ss;
        ss << year << "/" << setw(2) << setfill('0') << month << "/" << setw(2) << setfill('0') << targetDay;
        return ss.str();
    }
    
    int main() {
        int a, b, c, y1, y2;
        cin >> a >> b >> c >> y1 >> y2;
    
        for (int year = y1; year <= y2; ++year) {
            string result = calculateDate(year, a, b, c);
            cout << result << endl;
        }
    
        return 0;
    }    
\end{lstlisting}