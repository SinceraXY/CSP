\section{201612-1 中间数}

\subsection{题目背景}

\subsection{问题描述}

\subsection{输入格式}

\subsection{输出格式}

\subsection{样例 1 输入}

\subsection{样例 1 输出}

\subsection{样例解释}

\subsection{评测用例规模与约定}

\subsection{对讲义中代码的点评或纠错}
 
优点:

直观的算法: 算法直接遍历数组,对每个元素统计左右两侧较小和较大元素的数量,通过比较判断是否为平衡点。算法直观易懂。

使用 vector 动态数组: 使用了 vector 动态数组,提高了代码的灵活性和可维护性,无需显式指定数组的大小。

清晰的变量命名: 使用了具有描述性的变量名,例如 smaller 和 larger,有助于理解代码逻辑。

合理的条件判断: 在内层循环中使用 continue 跳过了与自身比较的情况,避免了不必要的计数。

\subsection{自己原创的代码的点评与注释}

首先对整数序列进行排序,然后找到排序后的中间位置的元素。接着,通过$count_if$函数分别统计小于和大于中间数的元素个数,最后比较这两个数量。如果相等,则存在中间数,输出中间数的值;否则,输出-1表示不存在中间数。

\begin{lstlisting}[language=C++]
    #include <iostream>
    #include <vector>
    #include <algorithm>
    
    using namespace std;
    
    int main() {
        // 读取输入的整数序列长度
        int n;
        cin >> n;
    
        // 读取整数序列
        vector<int> nums(n);
        for (int i = 0; i < n; ++i) {
            cin >> nums[i];
        }
    
        // 对整数序列进行排序
        sort(nums.begin(), nums.end());
    
        // 判断中间数是否存在
        int middleIndex = n / 2; // 中间数的下标
        int middleValue = nums[middleIndex]; // 中间数的值
        int countLess = count_if(nums.begin(), nums.end(), [&](int x) { return x < middleValue; });
        int countGreater = count_if(nums.begin(), nums.end(), [&](int x) { return x > middleValue; });
    
        // 输出结果
        if (countLess == countGreater) {
            cout << middleValue << endl; // 存在中间数,输出中间数的值
        } else {
            cout << -1 << endl; // 不存在中间数,输出-1
        }
    
        return 0;
    }    
\end{lstlisting}