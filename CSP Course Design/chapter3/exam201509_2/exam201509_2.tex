\section{201509-2 日期计算}

\subsection{题目背景}

\subsection{问题描述}

\subsection{输入格式}

\subsection{输出格式}

\subsection{样例 1 输入}

\subsection{样例 1 输出}

\subsection{样例解释}

\subsection{评测用例规模与约定}

\subsection{对讲义中代码的点评或纠错}
 
优点:

简洁明了的逻辑: 代码逻辑相对简单,清晰易懂。通过判断是否为闰年,然后根据每个月的天数计算月份和天数。

使用数组存储月份天数: 使用数组$normal$存储每个月的天数,提高了代码的可读性和维护性。

使用布尔类型表示是否为闰年: 使用$leap\_year$布尔变量表示是否为闰年,增加了代码的可读性。

缺点和改进空间:

未处理输入异常: 代码没有对输入进行异常处理。如果输入不符合预期,可能导致程序行为不可预测。建议添加一些输入验证机制,确保程序稳健性。

\subsection{自己原创的代码的点评与注释}

编写一个函数 isLeapYear 判断是否为闰年,根据题目中的条件进行判断。

编写一个函数 calculateDate 根据输入的年份和天数计算月份和日期。

在 main 函数中读取输入的年份和天数,调用 calculateDate 函数计算结果,最后输出月份和日期。

\begin{lstlisting}[language=C++]
    #include <iostream>
    using namespace std;
    // 判断是否为闰年的函数
    bool isLeapYear(int year) {
        return (year % 4 == 0 && year % 100 != 0) || (year % 400 == 0);
    }
    // 计算某年某一天是几月几日的函数
    void calculateDate(int year, int dayOfYear, int &month, int &day) {
        // 定义每个月的天数
        int daysInMonth[] = {0, 31, 28, 31, 30, 31, 30, 31, 31, 30, 31, 30, 31};
    
        // 如果是闰年,2月的天数为29
        if (isLeapYear(year)) {
            daysInMonth[2] = 29;
        }
        // 初始化月份和日期
        month = 1;
        day = 0;
        // 循环减去每个月的天数,直到剩余天数小于等于当前月的天数
        while (dayOfYear > daysInMonth[month]) {
            dayOfYear -= daysInMonth[month];
            month++;
        }
        // 剩余的天数即为日期
        day = dayOfYear;
    }
    int main() {
        // 输入年份和天数
        int year, dayOfYear;
        cin >> year >> dayOfYear;
    
        // 计算月份和日期
        int month, day;
        calculateDate(year, dayOfYear, month, day);
    
        // 输出结果
        cout << month << endl;
        cout << day << endl;
    
        return 0;
    }    
\end{lstlisting}