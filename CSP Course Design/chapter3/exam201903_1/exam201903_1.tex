\section{201903-1 小中大}

\subsection{题目背景}
在数据分析中,最小值最大值以及中位数是常用的统计信息。

\subsection{题目描述}%题目描述请保留在文档内


\subsection{输入格式}

输入的第一行包含一个整数$n$,表示商店的数量。

第二行包含$n$个整数,依次表示每个商店第一天的菜价。

\subsection{输出格式}

输出一行,包含$n$个正整数,依次表示每个商店第二天的菜价。

\subsection{样例输入}

\begin{lstlisting}[numbers=none]
8
4 1 3 1 6 5 17 9
\end{lstlisting}

\subsection{样例输出}

\begin{lstlisting}[numbers=none]
2 2 1 3 4 9 10 13
\end{lstlisting}

\subsection{评测用例规模与约定}

对于所有评测用例,$2\leq n\leq1000$,第一天每个商店的菜价为不超过 10000 的正整数。

\subsection{对讲义中代码的点评或纠错}
逻辑清晰:代码逻辑相对清晰,通过使用标准库函数$std::max\_element$和$std::min\_element$来查找最大值和最小值,以及计算中位数。

使用标准库:代码充分利用了C++标准库中的向量容器和算法函数,提高了代码的可读性和可维护性。

中位数计算方式:在计算中位数的时候,有一处可能会引发问题。具体来说,当中位数是一个整数时,使用$static\_cast<int>(median)$来截断小数部分是合理的。但在中位数是一个浮点数时,使用$fixed$和$setprecision$来格式化输出可能更好。

输入数据未验证:代码没有对输入数据的合法性进行验证,如果输入的数据不符合要求,可能会导致程序运行错误。建议添加输入数据的验证。
\subsection{自己原创的代码的点评与注释}

首先读入整数n和有序的测量数据,然后计算最大值、中位数和最小值,并按照从大到小的顺序输出这三个值。其中,处理中位数的逻辑根据测量数据的个数是奇数还是偶数进行了不同的分支处理。

\begin{lstlisting}[language=C++]
    #include <iostream>
    #include <vector>
    #include <cmath>  // 用于处理四舍五入
    
    using namespace std;
    
    int main() {
        // 读入数据
        int n;
        cin >> n;
        // 读入有序的测量数据
        vector<int> measurements(n);
        for (int i = 0; i < n; ++i) {
            cin >> measurements[i];
        }
        // 计算最大值、中位数和最小值
        int max_value = measurements[n - 1];  // 最大值即为最后一个元素
        int min_value = measurements[0];      // 最小值即为第一个元素
        int median_index = n / 2;             // 中位数的索引
    
        // 若n为奇数,则中位数即为中间的数;若n为偶数,则中位数为中间两个数的平均值
        int median_value = (n % 2 == 1) ? measurements[median_index] :
                                          round((measurements[median_index - 1] + measurements[median_index]) / 2.0);
        // 输出结果,按照从大到小的顺序输出
        cout << max_value << " " << median_value << " " << min_value << endl;
    
        return 0;
    }    
\end{lstlisting}