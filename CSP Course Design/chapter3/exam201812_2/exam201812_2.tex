\section{201812-2 小明放学}

\subsection{题目背景}

汉东省政法大学附属中学所在的光明区最近实施了名为“智慧光明”的智慧城市项目。具体到交通领域,通过“智慧光明”终端,可以看到光明区所有红绿灯此时此刻的状态。小明的学校也安装了
“智慧光明”终端,小明想利用这个终端给出的信息,估算自己放学回到家的时间。

\subsection{问题描述}

一次放学的时候,小明已经规划好了自己回家的路线,并且能够预测经过各个路段的时间。同时,小明通过学校里安装的“智慧光明”终端,看到了出发时刻路上经过的所有红绿灯的指示状态。请帮忙计算小明此次回家所需要的时间。

\subsection{输入格式}

输入的第一行包含空格分隔的三个正整数 $r、y、g$,表示红绿灯的设置。这三个数均不超过$10^6$。

输入的第二行包含一个正整数$n$,表示小明总共经过的道路段数和路过的红绿灯数目。

接下来的$n$行,每行包含空格分隔的两个整数$k、t$。$k = 0$ 表示经过了一段道路,将会耗时 $t$ 秒,此处 $t$ 不超过 $10^6$;$k$ = 1、2、3 时,分别表示出发时刻,此处的红绿灯状态是红灯、黄灯、绿灯,且倒计时显示牌上显示的数字是 $t$,此处 t 分别不会超过 $r、y、g$。

\subsection{输出格式}

输出一个数字,表示此次小明放学回家所用的时间。

\subsection{样例 1 输入}

\begin{lstlisting}[numbers=none]
    30 3 30
    8
    0 10
    1 5
    0 11
    2 2
    0 6
    0 3
    3 10
    0 3
\end{lstlisting}

\subsection{样例 1 输出}

\begin{lstlisting}[numbers=none]
46
\end{lstlisting}

\subsection{样例解释}

小明先经过第一段路,用时 10 秒。第一盏红绿灯出发时是红灯,还剩 5 秒;小明到达路口时,这个红绿灯已经变为绿灯,不用等待直接通过。接下来经过第二段路,用时 11 秒。第二盏红绿灯出发时是黄灯,还剩两秒;小明到达路口时,这个红绿灯已经变为红灯,还剩 11 秒。接下来经过第三、第四段路,用时 9 秒。第三盏红绿灯出发时是绿灯,还剩 10 秒;小明到达路口时,这个红绿灯已经变为红灯,还剩两秒。接下来经过最后一段路,用时 3 秒。共计 10 + 11 + 11 + 9 + 2 + 3 = 46 秒。

\subsection{评测用例规模与约定}

有些测试点具有特殊的性质:前 2 个测试点中不存在任何信号灯。测试点的输入数据规模:前6个测试点保证$n ≤ 10^3$;所有测试点保证$n ≤ 10^5$。

\subsection{对讲义中代码的点评或纠错}
 
功能实现简单: 代码实现了模拟红绿灯的功能,逻辑相对清晰,容易理解。

变量命名清晰: 变量命名相对清晰,例如$r、y、g$表示红、黄、绿灯的时间,n表示输入的次数等。

未考虑非法输入的情况: 没有处理可能的非法输入情况,比如负数的时间或无效的灯类型。

\subsection{自己原创的代码的点评与注释}

根据输入的每一段路的类型(道路或红绿灯),累加小明经过的时间。对于红绿灯,根据其状态和倒计时时间,计算小明需要等待的时间。

\begin{lstlisting}[language=C++]
    #include <iostream>

    using namespace std;
    
    int main() {
        // 红绿灯的设置
        int r, y, g;
        cin >> r >> y >> g;
    
        // 小明总共经过的道路段数和路过的红绿灯数目
        int n;
        cin >> n;
    
        // 初始化总时间
        int total_time = 0;
    
        // 遍历每一段路
        for (int i = 0; i < n; ++i) {
            int k, t;
            cin >> k >> t;
    
            if (k == 0) {
                // 如果是道路,直接加上时间
                total_time += t;
            } else {
                // 如果是红绿灯
                if (k == 1) {
                    // 红灯,需要等待剩余时间
                    total_time += t;
                } else if (k == 2) {
                    // 黄灯,需要等待剩余时间,同时加上红灯的时间
                    total_time += t + r;
                } else {
                    // 绿灯,直接通过
                }
            }
        }
    
        cout << total_time << endl;
    
        return 0;
    }    
\end{lstlisting}