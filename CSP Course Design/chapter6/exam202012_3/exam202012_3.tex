\section{202012-3 带配额的文件系统}

\subsection{题目背景}

\subsection{问题描述}

\subsection{输入格式}

\subsection{输出格式}

\subsection{样例 1 输入}

\subsection{样例 1 输出}

\subsection{样例解释}

\subsection{评测用例规模与约定}

\subsection{对讲义中代码的点评或纠错}
 
优点:

使用了面向对象的设计:通过定义结构体 node 表示文件系统中的节点,包含名称、孩子节点、父节点、文件配额等信息,使用面向对象的思想,提高了代码的可维护性和可扩展性。

使用了动态内存分配:通过使用 new 运算符创建节点,灵活地管理内存,避免了静态内存分配的限制,使得程序更加灵活。

使用了函数和模块化设计:通过将不同功能划分为不同的函数(opec、oper、opeq)和结构体成员函数,实现了代码的模块化设计,提高了代码的可读性和可维护性。

考虑了文件系统的多种操作:程序支持创建、删除文件和目录,以及查询文件和目录的配额信息,覆盖了文件系统的基本操作。

使用了 switch 语句:通过使用 switch 语句处理不同的操作类型,提高了代码的清晰度和可读性。

考虑了根节点的特殊情况:程序在处理根节点时进行了特殊处理,保证了对根节点的操作正确性。

考虑了目录的后代文件配额:在操作普通文件时,考虑了目录及其所有祖先结点的后代文件配额是否满足的问题,保证了数据的一致性。

使用了合适的数据结构:通过使用 vector 存储文件路径,使得路径操作更加灵活。

\subsection{你推荐的代码}

\href{202012-3 带配额的文件系统}{https://blog.csdn.net/qq\_43381135/article/details/112953778}

\subsection{网上的代码}
\begin{lstlisting}[language=C++]
    #include<iostream>
    #include<cstdio>
    #include<map>
    #include<vector>
    #include<cstring>
    #pragma warning (disable:4996)
    using namespace std;
    int num = 0;
    struct node {
        int father; //父节点
        map<string, int>child; //孩子节点
        int type; //1:文件   2:目录
        long long ld, lr;
        long long size;
        long long ld_r, lr_r;
    }Node[4000010];
    vector<pair<int, string> >reback;
    void Reback()
    {
        int i;
        for (i = 0; i < reback.size(); i++)
        {
            int a = reback[i].first;
            string b = reback[i].second;
            Node[a].child.erase(Node[a].child.find(b));
        }
    }
    string executeC()
    {
        int i, j;
        string filepath;
        long long filesize;
        cin >> filepath >> filesize;     //输入普通文件路径以及普通文件大小
        int last = -1;
        for (i = filepath.length() - 1; i >= 0; i--)
        {
            if (filepath[i] == '/')
            {
                last = i;       //找到最后一个/的位置,后面就是文件,前面就是目录
                break;
            }
        }
        int current = 1;
        int id = 0;
        reback.clear();
        int oldnum = num; 
        //前面的目录操作
        while (current < last)
        {
            string t = "";
            while (current < last && filepath[current] != '/')  //读入每一个目录文件的名字
            {
                t = t + filepath[current];
                current++;
            }
            current++;
            if (Node[id].child.find(t) == Node[id].child.end()) //没有这个目录,那就创建一个咯
            {
                num++;
                Node[id].child[t] = num;
                Node[num].father = id;
                Node[num].type = 2;
                Node[num].ld = LLONG_MAX / 3;
                Node[num].lr = LLONG_MAX / 3;
                reback.push_back(make_pair(id, t));
                id = num;
            }
            else                 //存在这个目录文件
            {
                int childid = Node[id].child[t];
                if (Node[childid].type == 1)   //如果这个是文件,则直接输出'N'
                {
                    num = oldnum;
                    Reback();
                    return "N";
                }
                id = childid;
            }
        }
    
        //最后一个的文件操作
        string t = "";
        for (i = last + 1; i < filepath.length(); i++)
        {
            t = t + filepath[i];
        }
        if (Node[id].child.find(t) != Node[id].child.end()) //存在这个文件
        {
            int childid = Node[id].child[t];
            if (Node[childid].type == 2)   //如果这个是目录,直接输出'N'
            {
                num = oldnum;
                Reback();
                return "N";
            }
        }
        //判断新的改变是否满足配额要求
        long long changesize = 0;
        if (Node[id].child.find(t) == Node[id].child.end()) changesize = filesize;
        else changesize = -Node[Node[id].child[t]].size + filesize;
        if (Node[id].ld_r + changesize > Node[id].ld)
        {
            num = oldnum;
            Reback();
            return "N";
        }
        int now = id;
        while (now != -1)
        {
            if (Node[now].lr_r + changesize > Node[now].lr)
            {
                id = oldnum;
                Reback();
                return "N";
            }
            now = Node[now].father;
        }
        //如果满足配额要求,则执行创建普通文件操作以及修改变化的参数
        if (Node[id].child.find(t) == Node[id].child.end())
        {
            num++;
            Node[num].type = 1;
            Node[num].father = id;
            Node[num].size = filesize;
            Node[id].child[t] = num;
        }
        else Node[Node[id].child[t]].size = filesize;
        Node[id].ld_r = Node[id].ld_r + changesize;
        now = id;
        while (now != -1)
        {
            Node[now].lr_r = Node[now].lr_r + changesize;
            now = Node[now].father;
        }
        return "Y";
    }
    string executeR()
    {
        int i;
        string filepath;
        cin >> filepath;          //输入文件路径
        int current = 1;
        int id = 0;
        int last = -1;
        for (i = filepath.length() - 1; i >= 0; i--)
        {
            if (filepath[i] == '/')
            {
                last = i;
                break;
            }
        }
        while (current < last)
        {
            string t = "";
            while (current < last && filepath[current] != '/')
            {
                t = t + filepath[current];
                current++;
            }
            current++;
            if (Node[id].child.find(t) == Node[id].child.end())  //没有这个目录,直接按照删除成功处理
            {
                return "Y";
            }
            else      //存在这个目录,继续扫描下去
            {
                int childid = Node[id].child[t];
                if (Node[childid].type == 1) return "Y";
                id = childid;
            }
        }
        string t = "";
        for (i = last + 1; i < filepath.length(); i++)   //提取最后想要删除的文件名
            t = t + filepath[i];
        if (Node[id].child.find(t) == Node[id].child.end())
            return "Y";
        long long delsize = 0;
        int delNode = Node[id].child[t];     //删除文件的大小
        if (Node[delNode].type == 1)
        {
            Node[id].ld_r = Node[id].ld_r - Node[delNode].size;
            delsize = Node[delNode].size;
            Node[id].child.erase(Node[id].child.find(t));
        }
        else if (Node[delNode].type == 2)
        {
            delsize = Node[delNode].lr_r;
            Node[id].child.erase(Node[id].child.find(t));
        }
        int now = id;
        while (now != -1)
        {
            Node[now].lr_r = Node[now].lr_r - delsize;
            now = Node[now].father;
        }
        return  "Y";
    }
    string executeQ()
    {
        int i;
        string filepath;
        cin >> filepath;
        long long ld, lr;
        cin >> ld >> lr;
        if (ld == 0) ld = LLONG_MAX / 3;
        if (lr == 0) lr = LLONG_MAX / 3;
        int id = 0;
        int last = -1;
        int current = 1;
        for (i = filepath.length() - 1; i >= 0; i--)
        {
            if (filepath[i] == '/')
            {
                last = i;
                break;
            }
        }
        while (current < last)
        {
            string t = "";
            while (current < last && filepath[current] != '/')
            {
                t = t + filepath[current];
                current++;
            }
            current++;
            if (Node[id].child.find(t) == Node[id].child.end())
                return "N";
            else
            {
                int childid = Node[id].child[t];
                if (Node[childid].type == 1) return "N";
                id = childid;
            }
        }
        string t = "";
        for (i = last + 1; i < filepath.length(); i++) t = t + filepath[i];
        int Qnode = 0;
        if (t == "") Qnode = 0;
        else
        {
            if (Node[id].child.find(t) == Node[id].child.end()) return "N";
            else Qnode = Node[id].child[t];
        }
        if (Node[Qnode].type == 1) return "N";
        if (ld < Node[Qnode].ld_r || lr < Node[Qnode].lr_r) return "N";
        Node[Qnode].ld = ld;
        Node[Qnode].lr = lr;
        return "Y";
    }
    int main()
    {
        freopen("in.in", "r", stdin);
        freopen("out.out", "w", stdout);
        int i, j;
        Node[0].type = 2;
        Node[0].ld = LLONG_MAX / 3;
        Node[0].lr = LLONG_MAX / 3;
        Node[0].father = -1;
        int op_num;
        cin >> op_num;   // 输入操作数
        for (i = 1; i <= op_num; i++)
        {
            char op;
            cin >> op;        //输入操作类型
            if (op == 'C')
            {
                cout << executeC() << endl;       //执行创建普通文件操作
            }
            else if (op == 'R')
            {
                cout << executeR() << endl;       //执行移除文件操作
            }
            else if (op == 'Q')
            {
                cout << executeQ() << endl;       //执行设置配额值操作
            }
        }
        return 0;
    }
    
\end{lstlisting}