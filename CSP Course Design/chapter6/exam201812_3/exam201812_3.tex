\section{201812-3 CIDR合并(vector struct分割)}

\subsection{题目背景}

\subsection{问题描述}

\subsection{输入格式}

\subsection{输出格式}

\subsection{样例 1 输入}

\subsection{样例 1 输出}

\subsection{样例解释}

\subsection{评测用例规模与约定}

\subsection{对讲义中代码的点评或纠错}
 
优点:

模块化设计: 代码通过结构体、函数和排序等模块化设计,使得不同功能的代码块独立存在,提高了代码的可读性和可维护性。

使用结构体: 结构体 Node 封装了IP地址和前缀长度,重载了小于操作符,有助于按照指定规则进行排序。

使用标准库: 代码使用了C++标准库提供的数据结构和算法,如 vector 和 sort,提高了代码的可靠性和性能。

合理的注释: 代码中包含了一些注释,解释了关键步骤和函数的作用,提高了代码的可读性。

输入处理: 通过 split 函数对IP地址进行切割,提取出需要的信息,并通过 add 函数将其添加到节点列表中,提高了代码的灵活性。

合并算法: merge 函数实现了合并算法,有效地将CIDR前缀合并,减少了冗余。

\subsection{你推荐的代码}

\href{201812-3 CIDR合并}{https://blog.csdn.net/wingrez/article/details/86601755}

\subsection{网上的代码}
\begin{lstlisting}[language=C++]
    #include<stdio.h>
    #include<stdlib.h>
    #include<string.h>
    #include<algorithm> 
    #define MAXN 1000005
    using namespace std;
    typedef unsigned int UI;
    
    struct IP{
        UI a[5];
        UI val;
    };
    
    int n;
    IP ip[MAXN];
    
    bool cmp(const IP &ip1, const IP &ip2){
        for(int i=0;i<5;i++){
            if(ip1.a[i]!=ip2.a[i]) return ip1.a[i]<ip2.a[i];
        }
        return false;
    }
    
    //将字符串转换为点分十进制形式 
    void deal_str(int no, char *str){
        int pn=0,xn=0;//点的数量,斜杠的数量 
        for(int i=0;i<strlen(str);i++){
            if(str[i]=='.') pn++;
            else if(str[i]=='/') xn++; 
        }
        
        //初始化
        for(int i=0;i<5;i++){
            ip[no].a[i]=0;
        }
        
        //处理字符串 
        int cnt=0;
        char *p1=str, *p2;
        char tmp[5];
        for(int i=0;i<pn;i++){
            p2=strchr(p1,'.');
            strncpy(tmp,p1,p2-p1);
            tmp[p2-p1]=0;
            ip[no].a[cnt++]=atoi(tmp);
            p1=p2+1;
        }
        
        if(xn>0){
            p2=strchr(p1,'/');
            strncpy(tmp,p1,p2-p1);
            tmp[p2-p1]=0;
            ip[no].a[cnt++]=atoi(tmp);
            p1=p2+1;
            strcpy(tmp,p1);
            ip[no].a[4]=atoi(tmp); 
        }
        else{
            strcpy(tmp,p1);
            ip[no].a[cnt++]=atoi(tmp);
            ip[no].a[4]=cnt*8; 
        }
    }
    
    void countVal(int no){ //计算十进制值 
        ip[no].val=0;
        for(int i=0;i<4;i++){
            ip[no].val+=ip[no].a[i]<<8*(3-i);
        }
    }
    
    void range(IP &ip,UI &l, UI &r){ //计算IP前缀能表示的数值范围 
        l=ip.val>>(32-ip.a[4])<<(32-ip.a[4]);
        r=ip.val|((1ull<<(32-ip.a[4]))-1);
    }
    
    void union1(){ //从小到大合并 
        int p=0;
        UI la,ra,lb,rb;
        for(int i=1;i<n;i++){
            range(ip[p],la,ra);
            range(ip[i],lb,rb);
            if(!(la<=lb && rb<=ra)) ip[++p]=ip[i];
        }
        n=p+1; 
    }
    
    bool judge_union(IP &ip1, IP &ip2, IP &res){ //判断同级能否合并 
        if(ip1.a[4]!=ip2.a[4]) return false;
        res=ip1; res.a[4]-=1;
        if(res.a[4]<0) return false;
        UI la,ra,lb,rb,laa,raa;
        range(ip1,la,ra);
        range(ip2,lb,rb);
        range(res,laa,raa);
        if(la==laa && rb==raa && lb<=ra+1) return true;
        return false;
    }
    
    void union2(){
        int p=0;
        IP res;
        for(int i=1;i<n;i++){
            if(judge_union(ip[p],ip[i],res)){
                ip[p]=res;
                while(p>0 && judge_union(ip[p-1], ip[p], res)) ip[--p]=res;
            }
            else ip[++p]=ip[i];
        }
        n=p+1;
    }
    
    int main(){
        char str[25];
        scanf("%d",&n);
        for(int i=0;i<n;i++){
            scanf("%s",str);
            deal_str(i,str);
            countVal(i);
    //		printf("%u.%u.%u.%u/%u\n",ip[i].a[0],ip[i].a[1],ip[i].a[2],ip[i].a[3],ip[i].a[4]);
    //		printf("%u\n",ip[i].val);
        }
        sort(ip,ip+n,cmp);
        union1();
        union2();
        for(int i=0;i<n;i++){
            printf("%u.%u.%u.%u/%u\n",ip[i].a[0],ip[i].a[1],ip[i].a[2],ip[i].a[3],ip[i].a[4]);
        } 
        return 0;
    }    
\end{lstlisting}