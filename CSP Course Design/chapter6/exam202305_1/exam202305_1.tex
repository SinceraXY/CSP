\section{202305-1 重复局面}

\subsection{题目背景}

\subsection{问题描述}

\subsection{输入格式}

\subsection{输出格式}

\subsection{样例 1 输入}

\subsection{样例 1 输出}

\subsection{样例解释}

\subsection{评测用例规模与约定}

\subsection{对讲义中代码的点评或纠错}
 
优点:

简洁明了: 代码结构清晰,逻辑简单,容易理解。主要通过$unordered\_map$来记录棋盘局面及其出现的次数。

使用STL容器: 使用了C++标准库提供的$unordered\_map$和vector容器,提高了代码的可读性和易用性。

遵循命名规范: 变量和函数命名符合常见的命名规范,提高了代码的可读性。

输入处理: 通过循环读取每个步骤的局面描述,实现了对不同步骤的处理。

\subsection{自己原创的代码的点评与注释}

使用二维数组positions存储每步棋后的局面。

使用$unordered\_map$统计每个局面出现的次数,其中键为局面的字符串表示,值为出现的次数。

遍历每步棋后的局面,更新局面出现次数并输出。

\begin{lstlisting}[language=C++]
    #include <iostream>
    #include <vector>
    #include <unordered_map>
    
    using namespace std;
    
    int main() {
        int n;
        cin >> n;
    
        vector<vector<string>> positions(n);  // 存储每步棋后的局面
        unordered_map<string, int> positionCount;  // 存储每个局面出现的次数
    
        // 读取每步棋后的局面
        for (int i = 0; i < n; ++i) {
            for (int j = 0; j < 8; ++j) {
                string row;
                cin >> row;
                positions[i].push_back(row);
            }
        }
    
        // 统计每个局面出现的次数
        for (int i = 0; i < n; ++i) {
            string currentPos;
            for (int j = 0; j < 8; ++j) {
                currentPos += positions[i][j];
            }
            positionCount[currentPos]++;
            cout << positionCount[currentPos] << endl;  // 输出当前局面是第几次出现
        }
    
        return 0;
    }    
\end{lstlisting}