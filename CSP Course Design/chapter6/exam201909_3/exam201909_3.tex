\section{201909-3 字符画(复杂模拟)}

\subsection{题目背景}

\subsection{问题描述}

\subsection{输入格式}

\subsection{输出格式}

\subsection{样例 1 输入}

\subsection{样例 1 输出}

\subsection{样例解释}

\subsection{评测用例规模与约定}

\subsection{对讲义中代码的点评或纠错}
 
优点:

模块化设计: 代码采用函数模块化设计,使得不同功能的代码块独立存在,提高了代码的可读性和可维护性。

采用结构体: 使用结构体 ColorRGB 封装了RGB颜色信息,提高了代码的结构性,便于维护和拓展。

使用常量定义: 使用常量 $MAX\_WIDTH$ 和 $MAX\_HEIGHT$ 定义了图像的最大宽度和高度,提高了代码的可读性和可维护性。

合理的注释: 代码中使用了注释,解释了一些关键的步骤和函数的作用,增强了代码的可读性。

灵活的颜色处理: 通过函数 outputBlockColor 实现了对颜色的灵活处理,支持颜色的切换和输出。

使用了标准库函数: 使用了标准库函数,如 cout 和 cin 进行输入输出,提高了代码的易用性。

\subsection{你推荐的代码}

\href{201909-3 字符画}{https://blog.csdn.net/best335/article/details/101296359}

\subsection{网上的代码}
\begin{lstlisting}[language=C++]
    #include<iostream>
    #include<cstring>
    #include<iomanip>
    #include<vector>
    using namespace std;
    unsigned char C[1080][1920][3];//C[n][m][Pixel:RGB] 表示原图片在第n行m列的像素颜色 
    inline unsigned char getPixel(const char&a,const char&b){//将16进制像素数转换为10进制的char 
        return char((isalpha(a)?(10+a-'a'):(a-'0'))*16+(isalpha(b)?(10+b-'a'):(b-'0')));
    }
    inline void outChar(const unsigned char&ch){//输出题意格式化的字符 
        cout<<"\\x"<<hex<<uppercase<<setw(2)<<int(ch);
    }
    inline void outStr(const string& str){//输出题意格式化的字符串 
        for(const char&c:str)outChar(c);
    }
    inline void outPixel(int i){//输出题意格式化的像素 
        vector<int> v;
        if(i==0)v.push_back(0);
        while(i>0) v.push_back(i%10),i/=10;//首先将数按位数分割 例:255 分割为 2、5、5三个数 
        for(i=v.size()-1;i>-1;--i) outChar(char('0'+v[i]));//输出每一位 
    }
    #include<fstream> 
    int main(){
        int m,n,p,q,_B;// W H 
        ifstream cin("C:\\Users\\Isidore\\Desktop\\out.txt");
        cin>>m>>n>>p>>q,_B=p*q,cout.fill('0');
        string s;
        for(int i=0;i<n;++i){
            for(int j=0;j<m;++j){
                cin>>s;
                switch(s.size()){//将s统一格式化为 #abcdef 
                    case 2://#a -> #aaaaaa 
                        s=s+string(5,s[1]);
                        break;
                    case 4://#abc -> #aabbcc
                        s="#"+string(2,s[1])+string(2,s[2])+string(2,s[3]);
                        break;
                }
                for(int k=0;k<3;++k) C[i][j][k]=getPixel(tolower(s[k+k+1]),tolower(s[k+k+2]));
            }
        }
        int R=0,G=0,B=0,r=0,g=0,b=0;
        for(int i=0;i<n;i+=q){//共n/q个字符块行 
            for(int j=0;j<m;j+=p){//每字符块行共m/p段 
                R=0,G=0,B=0;//以下处理属于i行j段的字符块
                for(int k=i,nk=k+q;k<nk;++k)
                    for(int l=j,nl=j+p;l<nl;++l)
                        R+=C[k][l][0],G+=C[k][l][1],B+=C[k][l][2];
                R/=_B,G/=_B,B/=_B;//求平均值 
                if(!(R==r&&G==g&&B==b)){//如果与该行上一段的颜色不同 
                    if(R==0&&G==0&&B==0)//如果与默认值相同 
                        outStr(string(1,char(27))+"[0m");
                    else//其他颜色处理 
                        outStr(string(1,char(27))+"[48;2;"),outPixel(R),outChar(';'),outPixel(G),outChar(';'),outPixel(B),outChar('m');
                    r=R,g=G,b=B;//记录上次的颜色 
                }
                outChar(' ');//输出 (n*m)/(p*q) 个空格 
            }
            if(R!=0||G!=0||B!=0) outStr(string(1,char(27))+"[0m");//行尾判断是否需要重置颜色 
            r=g=b=0;//重置默认颜色 
            outChar('\n');//输出n/q个回车 
        }
        return 0;
    }    
\end{lstlisting}