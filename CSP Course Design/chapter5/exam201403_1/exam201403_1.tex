\section{201403-1 相反数}

\subsection{题目背景}

\subsection{问题描述}

\subsection{输入格式}

\subsection{输出格式}

\subsection{样例 1 输入}

\subsection{样例 1 输出}

\subsection{样例解释}

\subsection{评测用例规模与约定}

\subsection{对讲义中代码的点评或纠错}
 
优点:

简洁高效: 代码逻辑相对简单,使用哈希集合 $unordered\_set$ 来实现对相反数的查找。这种方法能够在平均情况下实现 O(1) 的查找效率,从而提高了算法的效率。

使用了哈希集合: 利用哈希集合存储已经遍历过的数字,能够在常数时间内判断是否存在相反数,这在处理大规模数据时效率较高。

用 $unordered\_set$ 避免了排序: 使用哈希集合存储数字,不需要事先对数组进行排序,降低了时间复杂度。

避免了二重循环: 使用哈希集合的方式避免了传统的二重循环查找相反数,减少了算法的时间复杂度。

简单的计数方式: 使用一个变量 count 来计数相反数对的数量,逻辑清晰。

\subsection{自己原创的代码的点评与注释}

使用$unordered\_set$来存储整数,这样可以快速查找一个整数的相反数是否存在。遍历整数数组时,对于每个整数,只需查找其相反数是否在$unordered\_set$中,如果存在,则说明找到了一对相反数。最后,输出计数器的一半,因为每对相反数都被计数了两次。

\begin{lstlisting}[language=C++]

    #include <iostream>
    #include <unordered_set>
    #include <vector>
    
    using namespace std;
    
    int main() {
        int N;
        cin >> N;
        unordered_set<int> numbers;
        vector<int> input(N);
        for (int i = 0; i < N; ++i) {
            cin >> input[i];
            numbers.insert(input[i]);
        }
        int count = 0;
        for (int i = 0; i < N; ++i) {
            if (numbers.count(-input[i]) > 0) {
                count++;
            }
        }
        cout << count / 2 << endl;                // Each pair is counted twice (a and -a)
        return 0;
    }
\end{lstlisting}