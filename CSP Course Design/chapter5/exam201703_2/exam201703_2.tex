\section{201703-2 学生排队}

\subsection{题目背景}

\subsection{问题描述}

\subsection{输入格式}

\subsection{输出格式}

\subsection{样例 1 输入}

\subsection{样例 1 输出}

\subsection{样例解释}

\subsection{评测用例规模与约定}

\subsection{对讲义中代码的点评或纠错}
 
优点:

使用了链表容器:通过使用 std::list 链表容器,实现了高效的插入和移动操作,避免了数组的元素移动所带来的开销。

使用了算法库:通过使用 <algorithm> 头文件中的 find 函数,实现了在链表中查找元素的操作,提高了代码的简洁性和可读性。

使用了自动类型推导:在遍历链表元素时使用了 auto 关键字,提高了代码的简洁性。

使用了迭代器:通过使用迭代器对链表进行操作,提高了代码的可读性,并允许在链表中执行高效的插入和移动。

使用了逆向迭代器:在进行负数索引时,使用了 advance 函数和逆向迭代器,实现了从链表末尾开始查找的效果。

代码逻辑清晰:通过使用 std::list 提供的 splice 函数,代码逻辑相对清晰,易于理解。

\subsection{自己原创的代码的点评与注释}

用一个vector保存学生的学号队列,初始化为1到n。

对于每次调整,根据指令进行相应的操作。

向后移动:找到学号为p的学生位置,将其移出队列,然后插入到指定位置。

向前移动:找到学号为p的学生位置,将其移出队列,然后插入到指定位置。

输出最终的学号队列。

\begin{lstlisting}[language=C++]
    #include <iostream>
    #include <vector>
    using namespace std;
    int main() {
        int n, m;
        cin >> n; // 输入学生数量
        cin >> m; // 输入调整次数
    
        vector<int> students; // 保存学生的学号队列
    
        // 初始化学号队列
        for (int i = 1; i <= n; ++i) {
            students.push_back(i);
        }
    
        // 处理每次调整
        for (int i = 0; i < m; ++i) {
            int p, q;
            cin >> p >> q;
    
            // 向后移动
            if (q > 0) {
                vector<int>::iterator it = find(students.begin(), students.end(), p); // 找到学号为p的学生的位置
                int index = distance(students.begin(), it); // 获取位置索引
                students.erase(it); // 将学生移出队列
                students.insert(students.begin() + min(index + q, n), p); // 插入到指定位置
            }
            // 向前移动
            else if (q < 0) {
                vector<int>::iterator it = find(students.begin(), students.end(), p); // 找到学号为p的学生的位置
                int index = distance(students.begin(), it); // 获取位置索引
                students.erase(it); // 将学生移出队列
                students.insert(students.begin() + max(index + q, 0), p); // 插入到指定位置
            }
        }
    
        // 输出最终队列
        for (int i = 0; i < n; ++i) {
            cout << students[i] << " ";
        }
    
        return 0;
    }    
\end{lstlisting}