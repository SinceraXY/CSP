\section{201503-2 数字排序}

\subsection{题目背景}

\subsection{问题描述}

\subsection{输入格式}

\subsection{输出格式}

\subsection{样例 1 输入}

\subsection{样例 1 输出}

\subsection{样例解释}

\subsection{评测用例规模与约定}

\subsection{对讲义中代码的点评或纠错}
 
优点:

简洁的逻辑结构: 代码逻辑相对简单,易于理解。主要通过 map 统计每个整数的出现次数,然后使用 multimap 进行排序,最后遍历输出。

使用 STL 容器: 使用了 C++ 标准模板库(STL)中的 map 和 multimap 容器,能够方便地对整数及其出现次数进行管理、排序和输出。

使用 auto 关键字: 使用了 auto 关键字,简化了代码,减少了冗余的类型声明。

清晰的变量命名: 变量命名相对清晰,例如 $count\_map$ 表示整数出现次数的映射,$reversed\_count\_multimap$ 表示按次数降序排列的多重映射。

使用迭代器遍历容器: 使用了迭代器进行容器的遍历,提高了代码的可读性。

正确处理相同次数的整数: 使用 multimap 存储相同次数的整数,保证了在输出时能够按照整数升序排列。

使用 greater 比较器: 在实例化 multimap 时使用了 greater 比较器,确保了次数降序排列。

\subsection{自己原创的代码的点评与注释}

定义一个结构体 NumberCount 来存储整数及其出现的次数。结构体中包含整数 number 和次数 count。

使用 $unordered\_map$ 来统计每个整数出现的次数,其中键是整数值,值是对应的出现次数。

遍历输入的整数序列,更新 countMap 中每个整数的出现次数。

将 countMap 中的统计结果复制到一个 vector 中,每个元素是一个 NumberCount 结构体。

定义 NumberCount 结构体的比较函数,用于排序。排序规则是按出现次数递减排序,如果次数相同,则按整数值递增排序。

使用 std::sort 函数对 numbers 向量进行排序。

输出排序后的结果,即每个整数及其出现次数。

\begin{lstlisting}[language=C++]

    #include <iostream>
    #include <vector>
    #include <unordered_map>
    #include <algorithm>
    
    using namespace std;
    
    struct NumberCount {
        int number;
        int count;
    
        // 构造函数
        NumberCount(int num, int cnt) : number(num), count(cnt) {}
    
        // 用于排序的比较函数
        bool operator<(const NumberCount& other) const {
            // 按出现次数递减排序,如果次数相同,则按数值递增排序
            if (count != other.count) {
                return count > other.count;
            }
            return number < other.number;
        }
    };
    
    int main() {
        int n;
        cin >> n;
    
        unordered_map<int, int> countMap; // 用于统计每个整数出现的次数
        vector<NumberCount> numbers;      // 存储NumberCount结构体,便于排序
    
        for (int i = 0; i < n; ++i) {
            int num;
            cin >> num;
            countMap[num]++;
        }
    
        // 将统计结果存入NumberCount结构体中
        for (const auto& pair : countMap) {
            numbers.emplace_back(pair.first, pair.second);
        }
    
        // 对NumberCount结构体进行排序
        sort(numbers.begin(), numbers.end());
    
        // 输出结果
        for (const auto& numCount : numbers) {
            cout << numCount.number << " " << numCount.count << endl;
        }
    
        return 0;
    }    

\end{lstlisting}