\section{201412-1 门禁系统}

\subsection{题目背景}

\subsection{问题描述}

\subsection{输入格式}

\subsection{输出格式}

\subsection{样例 1 输入}

\subsection{样例 1 输出}

\subsection{样例解释}

\subsection{评测用例规模与约定}

\subsection{对讲义中代码的点评或纠错}
 
优点:

简洁明了: 代码逻辑相对简单,主要通过 $unordered\_map$ 实现了对读者编号出现次数的统计。整体结构清晰,易于理解。

使用了标准库容器: 使用了 C++ 标准库中的 $unordered\_map$ 容器,提供了高效的键值对存储和检索功能,不需要手动实现哈希表。

实时更新读者编号出现次数: 在遍历每一条记录时,实时更新对应读者编号在 $unordered\_map$ 中的次数,避免了额外的循环。

适用于大规模数据: 由于使用了哈希表,该算法在平均情况下具有 O(1) 的插入和检索效率,适用于处理大规模数据。

\subsection{自己原创的代码的点评与注释}

使用$unordered\_map$记录每位读者最后一次出现的位置。

对于每条记录,如果读者已经出现过,计算当前次数并更新记录,否则记录为1。

输出结果。

\begin{lstlisting}[language=C++]
    #include <iostream>
    #include <unordered_map>
    #include <vector>
    
    using namespace std;
    
    int main() {
        // 读取记录条数
        int n;
        cin >> n;
    
        // 用unordered_map记录每位读者最后一次出现的位置
        unordered_map<int, int> lastOccurrence;
    
        // 存储每条记录中的读者编号是第几次出现
        vector<int> occurrenceCount;
    
        // 处理每条记录
        for (int i = 0; i < n; ++i) {
            int reader;
            cin >> reader;
    
            // 如果读者已经出现过,更新记录为当前次数
            if (lastOccurrence.find(reader) != lastOccurrence.end()) {
                occurrenceCount.push_back(i - lastOccurrence[reader] + 1);
            } else {
                // 如果是第一次出现,记录为1
                occurrenceCount.push_back(1);
            }
    
            // 更新读者最后一次出现的位置
            lastOccurrence[reader] = i;
        }
    
        // 输出结果
        for (int i = 0; i < n; ++i) {
            cout << occurrenceCount[i];
            if (i < n - 1) {
                cout << " ";
            }
        }
    
        return 0;
    }    
\end{lstlisting}