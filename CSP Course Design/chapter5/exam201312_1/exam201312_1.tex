\section{201312-1 出现次数最多的数}

\subsection{题目背景}

\subsection{问题描述}

\subsection{输入格式}

\subsection{输出格式}

\subsection{样例 1 输入}

\subsection{样例 1 输出}

\subsection{样例解释}

\subsection{评测用例规模与约定}

\subsection{对讲义中代码的点评或纠错}
 
优点:

简洁的逻辑结构: 代码逻辑相对简单,易于理解。主要通过 map 统计每个整数的出现次数,然后遍历 map 找到出现次数最多的整数。

使用 STL 容器: 使用了 C++ 标准模板库(STL)中的 map 容器,能够方便地对整数及其出现次数进行管理和排序。

使用迭代器遍历容器: 使用了迭代器 countMap.begin(),能够更方便地遍历 map 容器中的元素。

考虑了相等次数的情况: 在找到出现次数最多的整数时,通过比较键值对的值和键来决定更新结果,考虑了出现次数相等的情况,选择键较小的整数。

合理的注释: 通过简单的注释,对代码中关键步骤和数据结构进行了说明,有助于代码的理解。

使用 auto 关键字: 使用了 auto 关键字,简化了代码,减少了冗余的类型声明。

清晰的变量命名: 变量命名相对清晰,例如 countMap 表示整数出现次数的映射,resultNum 表示结果整数,maxCount 表示最大出现次数。

错误处理: 对于键不存在的情况,map 会自动创建并初始化值为 0,避免了因为键不存在而导致的错误。

\subsection{自己原创的代码的点评与注释}

通过使用$unordered\_map$,我们可以有效地记录每个数字的出现次数。

遍历输入的数字,更新映射中对应数字的出现次数。

在遍历映射时,找到出现次数最多的数字。如果有多个数字出现次数相同,则选择最小的那个。

输出最终结果。

\begin{lstlisting}[language=C++]
    #include <iostream>
    #include <unordered_map>
    #include <climits>
    
    using namespace std;
    
    int main() {
        // 读取输入的 n
        int n;
        cin >> n;
    
        // 创建一个无序映射,用于记录每个数字的出现次数
        unordered_map<int, int> frequencyMap;
    
        // 读取 n 个数字,并在映射中更新它们的出现次数
        for (int i = 0; i < n; ++i) {
            int num;
            cin >> num;
            frequencyMap[num]++;
        }
    
        // 初始化最小的出现次数为最大整数
        int minFrequency = INT_MAX;
    
        // 初始化出现次数最多的数字为0
        int mostFrequentNumber = 0;
    
        // 遍历映射,找出出现次数最多的数字(如果有多个,选择最小的)
        for (const auto& entry : frequencyMap) {
            if (entry.second > minFrequency || (entry.second == minFrequency && entry.first > mostFrequentNumber)) {
                continue; // 如果当前数字的出现次数不大于当前最小出现次数,则跳过
            }
    
            // 更新最小出现次数和出现次数最多的数字
            minFrequency = entry.second;
            mostFrequentNumber = entry.first;
        }
    
        // 输出结果
        cout << mostFrequentNumber << endl;
    
        return 0;
    }    
\end{lstlisting}