\section{202206-2 寻宝!大冒险!}

\subsection{题目背景}

\subsection{问题描述}

\subsection{输入格式}

\subsection{输出格式}

\subsection{样例 1 输入}

\subsection{样例 1 输出}

\subsection{样例解释}

\subsection{评测用例规模与约定}

\subsection{对讲义中代码的点评或纠错}
 
优点:

使用了结构体和 pair:通过定义结构体 pair<int, int> 来表示树木的坐标,提高了代码的可读性和结构化程度。

使用了向量容器:通过使用 vector 容器来存储树木的坐标和藏宝图的信息,方便了对树木和藏宝图的动态管理,如添加、删除等操作。

使用了 find 算法:在检查树木的坐标是否存在于树木数组中时,使用了 find 算法,提高了代码的简洁性和可读性。

使用了静态转换:通过使用 $static\_cast$ 对布尔值进行静态类型转换,使得代码更加明确。

代码逻辑清晰:通过嵌套循环逐一检查树木的坐标是否与藏宝图相符,代码逻辑相对清晰,易于理解。

使用了初始化列表:在声明 $treasure\_map$ 时使用了初始化列表,提高了代码的简洁性和效率。

缺点和改进空间:

缺少输入验证:代码没有对输入的数据进行验证,特别是在输入向量大小和元素的情况下。如果输入的数据不符合要求,可能会导致程序运行错误。建议添加输入数据的验证。

\subsection{自己原创的代码的点评与注释}

读取输入数据,包括树的坐标、绿化图和藏宝图的大小。

使用二维向量存储树的坐标和藏宝图。

遍历绿化图上可能的左下角坐标,判断是否存在与藏宝图对应的坐标。

统计满足条件的坐标数量并输出结果。

\begin{lstlisting}[language=C++]
    #include <iostream>
    #include <vector>
    
    using namespace std;
    
    int main() {
        // 读入输入数据
        int n, L, S;
        cin >> n >> L >> S;
    
        // 用于记录树的坐标
        vector<pair<int, int>> treeCoordinates;
    
        // 读入树的坐标
        for (int i = 0; i < n; ++i) {
            int x, y;
            cin >> x >> y;
            treeCoordinates.push_back({x, y});
        }
    
        // 用于记录藏宝图
        vector<vector<int>> treasureMap(S + 1, vector<int>(S + 1));
    
        // 读入藏宝图
        for (int i = S; i >= 0; --i) {
            for (int j = 0; j <= S; ++j) {
                cin >> treasureMap[i][j];
            }
        }
    
        // 统计满足条件的坐标数量
        int count = 0;
    
        // 遍历绿化图上可能的左下角坐标
        for (int x = 0; x <= L - S; ++x) {
            for (int y = 0; y <= L - S; ++y) {
                // 判断是否满足条件
                bool valid = true;
                for (int i = 0; i <= S; ++i) {
                    for (int j = 0; j <= S; ++j) {
                        // 判断绿化图上的坐标是否与藏宝图对应
                        if (treasureMap[i][j] == 1 && treasureMap[i][j] != treeCoordinates[x + i].second) {
                            valid = false;
                            break;
                        }
                    }
                    if (!valid) {
                        break;
                    }
                }
    
                // 如果满足条件,则计数加一
                if (valid) {
                    ++count;
                }
            }
        }
    
        // 输出结果
        cout << count << endl;
    
        return 0;
    }    
\end{lstlisting}

\subsection{你推荐的代码}

\href{202206-2 寻宝!大冒险!}{https://blog.csdn.net/qq\_51800570/article/details/126673741}

\subsection{网上的代码}
\begin{lstlisting}[language=C++]
    #include <iostream>
    #include <cstring>
    #include <algorithm>
    #include <vector>
    
    using namespace std;
    typedef pair <int,int> PII;
    
    vector <PII> v;
    int m[60][60];
    int n, L, S;
    
    bool check(int x, int y)
    {
        //cout << endl;
        for (int i = S; i >= 0; i --)
        {
            for (int j = 0; j <= S; j ++)
            {
                int cx = S - i; //行的逆序,可以选择在存储的地方就改变,也可以在操作时计算原本的下标
                int cy = j;
                //cout << m[i][j] << " " << cx + x << " " << cy + y << endl;
    
                if (x + cx < 0 || x + cx > L || y + cy > L || y + cy < 0) return false; //如果小地图超出大地图的范围,停止
                bool flag = false;
                for (auto t : v)
                {
                    if (x + cx == t.first && y + cy == t.second)
                    {
                        flag = true;
                        break;
                    }
                }
                //cout << m[i][j] << " " << flag << " " << cx + x << " " << cy + y << endl;
                if ((m[i][j] && !flag) || (!m[i][j] && flag)) return false;
            }
        }
        //cout << 1;
        return true;
    }
    
    int main()
    {
        cin >> n >> L >> S;
        memset(m, 0, sizeof(m));
    
        for (int i = 0; i < n; i ++)
        {
            int x, y;
            cin >> x >> y;
            v.push_back({x, y});
            //cout << x << " " << y << endl;
        }
    
        for (int i = 0; i <= S; i ++)
        {
            for (int j = 0; j <= S; j ++)
            {
                cin >> m[i][j]; 
                //最先输入的是 B[S][0]⋯B[S][S] 一行,B[0][0]⋯B[0][S] 一行最后输入
                //不是正宗的数学坐标系啊,只是行的逆序罢了,这个地方卡了我一天,也是挺无语的
            }
        }
    
        int res = 0;
        for (int i = 0; i < n; i ++)
        {
            if (check(v[i].first, v[i].second))
            {
                res ++;
                //cout << v[i].first << " " << v[i].second << endl;
            }
        }
        cout << res;
        return 0;
    }    
\end{lstlisting}