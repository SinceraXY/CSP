\section{201403-2 窗口}

\subsection{题目背景}

\subsection{问题描述}

\subsection{输入格式}

\subsection{输出格式}

\subsection{样例 1 输入}

\subsection{样例 1 输出}

\subsection{样例解释}

\subsection{评测用例规模与约定}

\subsection{对讲义中代码的点评或纠错}
 
优点:

使用了结构体:通过定义结构体Window来表示窗口的位置和编号,提高了代码的可读性和结构化程度。

封装了判断点是否在窗口内的函数:使用isInside函数来判断给定的点是否在窗口内,提高了代码的模块化和可维护性。

使用了向量容器:通过使用vector容器来存储窗口信息,方便了对窗口的动态管理,如添加、删除等操作。

基本逻辑清晰:代码逻辑相对清晰,通过嵌套循环实现了窗口选择和置顶的操作。

\subsection{自己原创的代码的点评与注释}

首先定义了一个窗口结构体,包含窗口的坐标信息和顺序编号。然后,它依次读入窗口信息和鼠标点击信息,模拟点击过程并输出结果。在每次点击后,将被点击的窗口移到所有窗口的最顶层。程序使用一个循环遍历窗口来确定最顶层的窗口,并在点击后更新窗口的顺序。

\begin{lstlisting}[language=C++]
    #include <iostream>
    #include <vector>
    
    using namespace std;
    
    // 定义窗口结构体
    struct Window {
        int x1, y1, x2, y2; // 窗口左上角和右下角坐标
        int order;          // 窗口的顺序编号
    };
    
    // 判断点是否在窗口内部
    bool isInside(int x, int y, const Window& window) {
        return x >= window.x1 && x <= window.x2 && y >= window.y1 && y <= window.y2;
    }
    
    int main() {
        int N, M;
        cin >> N >> M;
    
        vector<Window> windows(N);
    
        // 读入窗口信息
        for (int i = 0; i < N; ++i) {
            cin >> windows[i].x1 >> windows[i].y1 >> windows[i].x2 >> windows[i].y2;
            windows[i].order = i + 1; // 窗口编号从1开始
        }
    
        // 模拟鼠标点击
        for (int i = 0; i < M; ++i) {
            int x, y;
            cin >> x >> y;
    
            bool clicked = false;
            // 从顶层到底层遍历窗口,找到最顶层的窗口
            for (int j = N - 1; j >= 0; --j) {
                if (isInside(x, y, windows[j])) {
                    cout << windows[j].order << endl;
                    // 将当前点击的窗口移至顶层
                    Window clickedWindow = windows[j];
                    windows.erase(windows.begin() + j);
                    windows.push_back(clickedWindow);
                    clicked = true;
                    break;
                }
            }
    
            // 如果没有窗口被点击,则输出IGNORED
            if (!clicked) {
                cout << "IGNORED" << endl;
            }
        }
    
        return 0;
    }    
\end{lstlisting}

\subsection{你推荐的代码}

\href{201403-2 窗口}{https://blog.csdn.net/tigerisland45/article/details/54782238}

\subsection{网上的代码}
\begin{lstlisting}[language=C++]
    #include <iostream>
     
    using namespace std;
     
    const int N = 10;
     
    struct {
        int winno;
        int x1, y1, x2, y2;
    } win[N];
     
    struct {
        int x, y;
    } point[N];
     
    int order[N];
     
    int main()
    {
        int n, m;
     
        // 输入数据
        cin >> n >> m;
        for(int i=0; i<n; i++) {
            win[i].winno = i + 1;
            cin >> win[i].x1 >> win[i].y1 >> win[i].x2 >> win[i].y2;
        }
        for(int i=0; i<m; i++)
            cin >> point[i].x >> point[i].y;
     
        // 初始化窗口显示顺序
        for(int i=0; i<n; i++)
            order[i] = n - i - 1;
     
        // 处理(模拟)鼠标点击
        int winno, temp;
        for(int i=0; i<m; i++) {
            winno = -1;     // 被点击的窗口号,-1表示未被点击
     
            // 判断哪个窗口被点击
            for(int j=0; j<n; j++) {
                if(win[order[j]].x1 <= point[i].x && point[i].x <= win[order[j]].x2 &&
                        win[order[j]].y1 <= point[i].y && point[i].y <= win[order[j]].y2) {
                    // 得到窗口号
                    winno = win[order[j]].winno;
     
                    // 将被点击的窗口移到最前端
                    temp = order[j];
                    for(int k=j; k>0; k--)
                        order[k] = order[k-1];
                    order[0] = temp;
     
                    break;
                }
            }
     
            // 输出结果
            if(winno == -1)
                cout << "IGNORED" << endl;
            else
                cout << winno << endl;
        }
     
        return 0;
    }
\end{lstlisting}