\section{201912-2 回收站选址}

\subsection{题目背景}

\subsection{问题描述}

\subsection{输入格式}

\subsection{输出格式}

\subsection{样例 1 输入}

\subsection{样例 1 输出}

\subsection{样例解释}

\subsection{评测用例规模与约定}

\subsection{对讲义中代码的点评或纠错}
 
优点:

使用了结构体:通过定义结构体Coordinate来表示坐标点的属性,提高了代码的可读性和结构化程度。

封装了函数:通过将垃圾检查和地址计数的逻辑封装成了两个函数hasGarbage和countAddresses,提高了代码的模块化和可维护性。

使用了向量容器:通过使用vector容器来存储垃圾的坐标,方便了对垃圾的动态管理,如添加、删除等操作。

代码逻辑清晰:通过循环遍历垃圾坐标,进行相应的检查和计数,代码逻辑相对清晰,易于理解。

使用了初始化列表:在Coordinate结构体的构造函数中使用了初始化列表,提高了代码的简洁性和效率。

使用了$emplace\_back$:在向garbage向量中添加元素时使用了$emplace\_back$,避免了不必要的拷贝构造,提高了性能。

缺点和改进空间:

没有进行输入数据的合法性验证:代码没有对输入的数据进行验证,如果输入的数据不符合要求,可能会导致程序运行错误。建议添加输入数据的验证。

\subsection{自己原创的代码的点评与注释}

用$unordered\_set$存储垃圾点坐标,以便快速查找。

遍历每个垃圾点,判断其上下左右四个邻居位置是否都存在垃圾。

计算四个对角位置中存在垃圾的个数,作为得分。

根据条件判断是否满足建站条件,满足则更新相应得分的回收站选址个数。

输出最终结果。

\begin{lstlisting}[language=C++]
    #include <iostream>
    #include <vector>
    #include <unordered_set>
    
    using namespace std;
    
    int main() {
        // 读取输入
        int n;
        cin >> n;
    
        // 垃圾点坐标存储在set中,便于查找
        unordered_set<int> garbage_points;
        
        // 读取垃圾点坐标并存储
        for (int i = 0; i < n; ++i) {
            int x, y;
            cin >> x >> y;
            garbage_points.insert(x * 1000000 + y);  // 将二维坐标转为一维,确保唯一性
        }
    
        // 初始化得分为0的回收站选址个数
        vector<int> scores(5, 0);
    
        // 遍历每个垃圾点,判断是否满足建站条件
        for (auto point : garbage_points) {
            int x = point / 1000000;
            int y = point % 1000000;
    
            // 上下左右四个邻居位置
            bool up = garbage_points.count((x, y + 1));
            bool down = garbage_points.count((x, y - 1));
            bool left = garbage_points.count((x - 1, y));
            bool right = garbage_points.count((x + 1, y));
    
            // 在 (x ± 1, y ± 1) 四个对角位置中有几处存在垃圾
            int diagonal_score = 0;
            diagonal_score += garbage_points.count((x - 1, y - 1));
            diagonal_score += garbage_points.count((x - 1, y + 1));
            diagonal_score += garbage_points.count((x + 1, y - 1));
            diagonal_score += garbage_points.count((x + 1, y + 1));
    
            // 判断是否满足建站条件
            if (up && down && left && right) {
                scores[diagonal_score]++;
            }
        }
    
        // 输出结果
        for (int score : scores) {
            cout << score << endl;
        }
    
        return 0;
    }    
\end{lstlisting}

\subsection{你推荐的代码}

\href{201912-2 回收站选址}{https://blog.csdn.net/weixin\_45884316/article/details/103604320}

\subsection{网上的代码}
\begin{lstlisting}[language=C++]
    #include<stdio.h>
    int main()
    {
        long long int a[1000][2];
        int b[1000]={0},c[5]={0};
        int n,i,j;
        scanf("%d",&n);
        for(i=0;i<n;i++)
            scanf("%lld%lld",&a[i][0],&a[i][1]);//数组的值为坐标 
        for(i=0;i<n;i++)
        {
            for(j=i+1;j<n;j++)
            {
                if((a[i][0]==a[j][0]&&a[i][1]==a[j][1]+1)||(a[i][0]==a[j][0]&&a[i][1]==a[j][1]-1)||	(a[i][0]==a[j][0]-1&&a[i][1]==a[j][1])||(a[i][0]==a[j][0]+1&&a[i][1]==a[j][1]))
                {
                    b[i]++;
                    b[j]++;
                }
            }
        }
        for(i=0;i<n;i++)
        {
            int count=0;//对角线上有几个 
            if(b[i]==4)
            {
                for(j=0;j<n;j++)
                {
                    if((a[i][0]==a[j][0]+1&&a[i][1]==a[j][1]+1)||(a[i][0]==a[j][0]+1&&a[i][1]==a[j][1]-1)||
                    (a[i][0]==a[j][0]-1&&a[i][1]==a[j][1]+1)||(a[i][0]==a[j][0]-1&&a[i][1]==a[j][1]-1))
                        count++;
                }
                c[count]++;
            } 
        }
        for(i=0;i<=4;i++)
            printf("%d\n",c[i]);
        return 0;
    }    
\end{lstlisting}