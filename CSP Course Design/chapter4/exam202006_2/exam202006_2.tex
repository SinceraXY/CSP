\section{202006-2 稀疏向量}

\subsection{题目背景}

\subsection{问题描述}

\subsection{输入格式}

\subsection{输出格式}

\subsection{样例 1 输入}

\subsection{样例 1 输出}

\subsection{样例解释}

\subsection{评测用例规模与约定}

\subsection{对讲义中代码的点评或纠错}
 
优点:

使用了 pair:通过使用 pair 存储键值对,提高了代码的可读性,使得代码中每个元素都有相关联的键。

使用了 vector 容器:通过使用 vector 存储键值对的集合,方便了对键值对的动态管理,如添加、删除等操作。

使用了 long long:在计算结果时,使用了 long long 类型来防止溢出,保证了计算的准确性。

代码逻辑清晰:通过使用两个指针(i 和 j)来遍历两个向量,并根据键的大小比较来计算乘积之和,代码逻辑相对清晰,易于理解。

采用了合理的循环控制:通过使用 while 循环,避免了写繁琐的 for 循环,提高了代码的简洁性。

\subsection{自己原创的代码的点评与注释}

通过 pair<int, int> 存储稀疏表示的向量,其中 first 表示维度的索引,second 表示在该维度上的非零值。

通过两个指针 i 和 j 分别遍历向量 u 和 v 的稀疏表示,比较当前维度的索引值。

根据比较结果更新内积的值。

输出最终计算得到的内积。

\begin{lstlisting}[language=C++]
    #include <iostream>
    #include <vector>
    
    using namespace std;
    
    int main() {
        // 读取输入数据
        int n, a, b;
        cin >> n >> a >> b;
    
        // 用于存储稀疏表示的向量
        vector<pair<int, int>> u_sparse(a);
        vector<pair<int, int>> v_sparse(b);
    
        // 读取向量 u 的稀疏表示
        for (int i = 0; i < a; ++i) {
            cin >> u_sparse[i].first >> u_sparse[i].second;
        }
    
        // 读取向量 v 的稀疏表示
        for (int i = 0; i < b; ++i) {
            cin >> v_sparse[i].first >> v_sparse[i].second;
        }
    
        // 计算内积
        int result = 0;
        int i = 0, j = 0;
    
        while (i < a && j < b) {
            // 如果两个稀疏向量在当前维度上有非零值
            if (u_sparse[i].first == v_sparse[j].first) {
                result += u_sparse[i].second * v_sparse[j].second;
                ++i;
                ++j;
            } else if (u_sparse[i].first < v_sparse[j].first) {
                // 如果向量 u 在当前维度上有非零值而向量 v 没有
                ++i;
            } else {
                // 如果向量 v 在当前维度上有非零值而向量 u 没有
                ++j;
            }
        }
    
        // 输出结果
        cout << result << endl;
    
        return 0;
    }    
\end{lstlisting}