\section{202305-2 矩阵运算}

\subsection{题目背景}

\subsection{问题描述}

\subsection{输入格式}

\subsection{输出格式}

\subsection{样例 1 输入}

\subsection{样例 1 输出}

\subsection{样例解释}

\subsection{评测用例规模与约定}

\subsection{对讲义中代码的点评或纠错}
 
优点:

使用了合适的数据结构:使用了二维向量(vector of vectors)来表示矩阵,这是一个在C++中灵活且方便的数据结构。

代码结构清晰:通过使用嵌套的循环,代码结构相对清晰,易于理解。

使用了适当的数据类型:在矩阵乘法过程中,使用了long long类型来处理可能的溢出问题,保证了计算结果的准确性。

缺点和改进空间:

未进行错误检查:代码没有对输入数据的合法性进行验证,如果输入的数据不符合要求,可能会导致程序运行错误。建议添加输入数据的验证。

\subsection{自己原创的代码的点评与注释}

通过嵌套的循环读取矩阵 Q、K、V 以及向量 W 的元素。

在计算结果时,利用嵌套循环遍历矩阵元素,按照题目描述的公式进行计算。

输出计算结果。

\begin{lstlisting}[language=C++]
    #include <iostream>
    #include <vector>
    using namespace std;
    int main() {
        // 读取输入数据
        int n, d;
        cin >> n >> d;
    
        // 读取矩阵 Q
        vector<vector<int>> Q(n, vector<int>(d));
        for (int i = 0; i < n; ++i) {
            for (int j = 0; j < d; ++j) {
                cin >> Q[i][j];
            }
        }
    
        // 读取矩阵 K
        vector<vector<int>> K(n, vector<int>(d));
        for (int i = 0; i < n; ++i) {
            for (int j = 0; j < d; ++j) {
                cin >> K[i][j];
            }
        }
    
        // 读取矩阵 V
        vector<vector<int>> V(n, vector<int>(d));
        for (int i = 0; i < n; ++i) {
            for (int j = 0; j < d; ++j) {
                cin >> V[i][j];
            }
        }
    
        // 读取向量 W
        vector<int> W(n);
        for (int i = 0; i < n; ++i) {
            cin >> W[i];
        }
    
        // 计算结果并输出
        for (int i = 0; i < n; ++i) {
            for (int j = 0; j < d; ++j) {
                // 计算 (Q × KT) 中第 i 行的每个元素与 W(i) 相乘的结果,并与 V 相乘
                cout << W[i] * (Q[i][0] * K[i][j] + Q[i][1] * K[i][j] + Q[i][2] * K[i][j]) * V[i][j] << " ";
            }
            cout << endl;
        }
    
        return 0;
    }    
\end{lstlisting}