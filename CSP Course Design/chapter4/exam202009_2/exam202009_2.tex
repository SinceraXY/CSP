\section{202009-2 风险人群筛查}

\subsection{题目背景}

\subsection{问题描述}

\subsection{输入格式}

\subsection{输出格式}

\subsection{样例 1 输入}

\subsection{样例 1 输出}

\subsection{样例解释}

\subsection{评测用例规模与约定}

\subsection{对讲义中代码的点评或纠错}
 
优点:

合理的数据结构选择: 使用了两个vector<int>来存储通过和滞留的人数,这是合理的选择,可以方便地对结果进行统计。

合理的循环嵌套结构: 使用了两层嵌套循环,外层遍历人,内层遍历时间,对每个人在每个时间点的位置进行判断。这样的结构较为清晰。

逻辑清晰: 通过布尔变量inside来表示人是否在矩形区域内,通过cnt来表示连续在矩形区域内的时间数,逻辑相对清晰。

合理的变量命名: 变量命名相对清晰,有助于理解代码的功能

缺点和改进空间:

未处理输入异常: 代码没有对输入进行异常处理。如果输入不符合预期,可能导致程序行为不可预测。建议添加一些输入验证机制,确保程序稳健性。

\subsection{自己原创的代码的点评与注释}

通过循环读取每个居民的位置记录。

遍历位置记录,判断每个时刻是否位于高危区域内。

统计经过高危区域的人数(有任意一个时刻在高危区域内即可)。

判断是否曾在高危区域逗留,需要连续 k 个或更多时刻在高危区域内。

\begin{lstlisting}[language=C++]
    #include <iostream>
    #include <vector>
    
    using namespace std;
    
    int main() {
        // 读取输入参数
        int n, k, t, xl, yd, xr, yu;
        cin >> n >> k >> t >> xl >> yd >> xr >> yu;
    
        // 记录经过高危区域和逗留的人数
        int passCount = 0, stayCount = 0;
    
        // 遍历每个居民
        for (int i = 0; i < n; ++i) {
            // 读取居民的位置记录
            vector<pair<int, int>> positions;
            for (int j = 0; j < t; ++j) {
                int x, y;
                cin >> x >> y;
                positions.push_back({x, y});
            }
    
            // 记录连续位于高危区域的次数
            int consecutiveCount = 0;
    
            // 遍历位置记录
            for (int j = 0; j < t; ++j) {
                // 判断是否位于高危区域
                if (positions[j].first >= xl && positions[j].first <= xr &&
                    positions[j].second >= yd && positions[j].second <= yu) {
                    // 位于高危区域,连续计数加一
                    consecutiveCount++;
                } else {
                    // 不位于高危区域,重置连续计数
                    consecutiveCount = 0;
                }
    
                // 判断是否满足逗留条件
                if (consecutiveCount >= k) {
                    stayCount++;
                    break; // 已经满足逗留条件,不再判断后续时刻
                }
            }
    
            // 判断是否经过高危区域
            if (consecutiveCount > 0) {
                passCount++;
            }
        }
    
        // 输出结果
        cout << passCount << endl;
        cout << stayCount << endl;
    
        return 0;
    }
\end{lstlisting}