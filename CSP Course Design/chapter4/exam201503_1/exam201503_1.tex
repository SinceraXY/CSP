\section{201503-1 图像旋转}

\subsection{题目背景}

\subsection{问题描述}

\subsection{输入格式}

\subsection{输出格式}

\subsection{样例 1 输入}

\subsection{样例 1 输出}

\subsection{样例解释}

\subsection{评测用例规模与约定}

\subsection{对讲义中代码的点评或纠错}
 
优点:

使用了标准库 vector: 代码使用了C++标准库中的 vector 来表示二维矩阵,这是一个合理的选择,方便了矩阵的动态创建和访问。

合理的输入循环: 使用了嵌套循环,逐行逐列读取输入矩阵的元素,使得输入过程比较清晰。

合理的矩阵旋转逻辑: 使用两个嵌套循环,从右向左遍历每一列,再从上到下遍历每一行,输出逆时针旋转后的结果。

缺点和改进空间:

未处理输入异常: 代码没有对输入进行异常处理。如果输入不符合预期,可能导致程序行为不可预测。建议添加一些输入验证机制,确保程序稳健性。

\subsection{自己原创的代码的点评与注释}

首先读取图像矩阵的行数和列数,然后读入图像矩阵。接着,它创建一个新的矩阵rotatedMatrix来存储逆时针旋转90度后的结果。最后,通过两层嵌套循环遍历原始矩阵,将每个元素按照旋转规则放置到新矩阵中。最终,输出旋转后的矩阵。

\begin{lstlisting}[language=C++]
    #include <iostream>
    #include <vector>
    
    using namespace std;
    
    int main() {
        // 读入图像矩阵的行数和列数
        int n, m;
        cin >> n >> m;
    
        // 读入图像矩阵
        vector<vector<int>> matrix(n, vector<int>(m));
        for (int i = 0; i < n; ++i) {
            for (int j = 0; j < m; ++j) {
                cin >> matrix[i][j];
            }
        }
    
        // 逆时针旋转矩阵
        // 注意:旋转后的行数变为原来的列数,列数变为原来的行数
        vector<vector<int>> rotatedMatrix(m, vector<int>(n));
    
        for (int i = 0; i < n; ++i) {
            for (int j = 0; j < m; ++j) {
                // 逆时针旋转 90 度后,原矩阵的第i行第j列元素,变为旋转后矩阵的第j行第(n-1-i)列元素
                rotatedMatrix[j][n - 1 - i] = matrix[i][j];
            }
        }
    
        // 输出旋转后的矩阵
        for (int i = 0; i < m; ++i) {
            for (int j = 0; j < n; ++j) {
                cout << rotatedMatrix[i][j] << " ";
            }
            cout << endl;
        }
    
        return 0;
    }    
\end{lstlisting}