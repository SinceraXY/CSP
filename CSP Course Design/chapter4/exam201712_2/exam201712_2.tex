\section{201712-2 游戏}

\subsection{题目背景}

\subsection{问题描述}

\subsection{输入格式}

\subsection{输出格式}

\subsection{样例 1 输入}

\subsection{样例 1 输出}

\subsection{样例解释}

\subsection{评测用例规模与约定}

\subsection{对讲义中代码的点评或纠错}
 
优点:

使用了标准库vector:使用了C++的标准库vector来存储每个人的编号,这是一个合适的选择,可以方便地进行插入和删除操作。

简洁的初始化:利用循环简洁地初始化了children向量,使其包含1到n的编号。

使用了合理的循环结构:使用了两个嵌套循环结构,其中外层循环迭代直到只剩下一个人,内层循环处理报数的逻辑。

缺点和改进空间:

不够高效:使用erase操作来删除vector中的元素,这涉及到移动后续元素,导致时间复杂度较高,特别是当k较大时。可以考虑使用链表等数据结构,以提高删除操作的效率。

未处理输入异常:代码没有对输入进行异常处理,如果输入不符合预期,可能导致程序行为不可预测。建议添加一些输入验证机制,确保程序稳健性。

\subsection{自己原创的代码的点评与注释}

使用一个数组来表示每个小朋友是否被淘汰。在循环中,按照规则报数,淘汰符合条件的小朋友,直到只剩下一个小朋友为止。最后输出获胜小朋友的编号。

\begin{lstlisting}[language=C++]
    #include <iostream>
    using namespace std;
    
    int main() {
        // 读取输入的 n 和 k
        int n, k;
        cin >> n >> k;
    
        // 定义一个数组表示小朋友是否被淘汰,初始都为 false
        bool eliminated[1001] = {false};
    
        // 初始化报数和当前小朋友编号
        int count = 0;
        int current = 1;
    
        // 当还有多于一个小朋友时进行循环
        while (count < n - 1) {
            // 如果当前小朋友没有被淘汰
            if (!eliminated[current]) {
                // 报数加1
                count++;
    
                // 如果报的数是 k 的倍数或者末位数为 k,则淘汰当前小朋友
                if (count % k == 0 || count % 10 == k) {
                    eliminated[current] = true;
                    count--;
                }
            }
    
            // 移动到下一个小朋友
            current = (current % n) + 1;
        }
    
        // 输出最后获胜的小朋友编号
        for (int i = 1; i <= n; i++) {
            if (!eliminated[i]) {
                cout << i << endl;
                break;
            }
        }
    
        return 0;
    }    
\end{lstlisting}